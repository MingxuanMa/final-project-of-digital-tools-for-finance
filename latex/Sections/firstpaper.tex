\subsection{Modern Portfolio Theory}
Mean-Variance optimization, better known as modern portfolio theory, describes the school of \citeauthor{markowitz_1952}, whereby the trade-off between risk and reward is solved, such that the expected return is maximized given a certain level of risk (cf \cite{markowitz_1952}). This mathematical approach to asset allocation is the formalization of diversification, which allows the portfolio to reduce all asset specific risk to the level that only market risk remains. This implies that a portfolio has a lower risk to the same expected return compared to a non optimally diversified strategy and therefore, arguably the most efficient way to asset allocation. However, this only holds true given the true expected returns and covariances are known, since those are crucial for an optimal allocation.\\
Historical correlations might indicate future ones but are not perfectly matching. Thus, distortions arise and portfolios are non-optimally diversified in future periods. This parameter uncertainty is a key problem to solve and was addressed in multiple papers, including the ones mentioned later in this chapter. Other scholars that criticised modern portfolio theory for its high sensitivity to input changes are \cite{chopra_1993}, that prove the issue mathematically; \cite{best_grauer_1991}, that assessed the impact of such on expected returns; \cite{jobson_korkie_1981}, that promote using sharpe ratios for optimizing mean-variance portfolios; and \cite{broadie_1981}, who discusses the overestimation of expected returns through varying estimation errors.