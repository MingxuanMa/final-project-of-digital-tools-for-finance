\subsection{\citetitle{anderson_cheng_2016}} \label{mainpaper}
One of the more sophisticated approaches is the one of \cite{anderson_cheng_2016}. They propose a Bayesian-averaging portfolio choice strategy to overcome the problem of parameter uncertainty.
\citeauthor{anderson_cheng_2016}'s method accounts for parameter uncertainty by using a Bayesian averaging approach over various models. In each period, they recalculate the covariance matrix and thereby optimal weights and reevaluate all historical models using Bayesian updating. The authors include robust mean-variance optimization by creating the optimization problem in a way that includes uncertainty about the correctness of a model. The advantage over rolling window approaches is that a model never fully dies out and no information is lost. Additionally, they allow investors to be in doubt that predictions are correct and thus, to focus on the worst scenario that is somewhat reasonable, given a range of predictions. They include a model parameter that lets them set the agents level of aversion against mis-specification; with the limit taking the normal form of the standard mean-variance problem.\\
Their proposed technique follows the following steps to arrive at their optimal weights in each period respectively:
\begin{enumerate}
  \item At date t-1, there are t-1 existing models which make mean and covariance predictions of excess returns on time t. Also, each model at date t-1 is assigned a prior probability.
  \item At date t, a new model is born. Therefore, the prior probabilities of all t models are adjusted for the inclusion of this new model. 
  \item The excess returns on date t are observed. The parameters of each model are updated using the new information. Prior probabilities are also updated using Bayesian Rules. Optimal portfolio choices at time t are computed. 
  \item At date t+1, the excess returns on t+1 are observed. Therefore, the excess portfolio returns are obtained by the product between the optimal portfolio choices at time t and the excess returns on t+1. 
\end{enumerate}
Important to note, that each model assumes constant mean and covariance for future periods. At each point in time, all the historical models get modified by using the new gained knowledge from this point in time. Each single one of them produces a new prediction at that point and the probability that the model is correct is calculated, based on the information known up to that day.\\
By following this sophisticated approach, \citeauthor{anderson_cheng_2016} produce robust predictions on future means and variances by accounting for parameter and model uncertainty, thus produce better estimations. See chapter \ref{methodology} of this paper for a detailed treatment of their, and our, methodology. During the empirical part of their paper, the authors show that their strategy yields superior results in real stock data and artificial return series, what we will be replicating with own data at a later stage.