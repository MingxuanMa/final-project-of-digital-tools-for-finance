\subsection{Resampling}
Among the practitioners that tackled the aforementioned problem was \citeauthor{michaud_1998}, who patented his Resampled Efficiency techniques (cf. \cite{michaud_1998}) in 1998. Just under a decade later, \cite{michaud_2007} extended this Resampled Efficiency solution formally, by showing that it is a Bayesian-based generalization of Markowitz’s approach. Their solution is based on Monte Carlo simulations to account for parameter uncertainty in mean-variance portfolio optimization.
In their later paper (\cite{michaud_2007}), the scholars argue that, while the theoretical idea behind Markovitz's portfolio optimization is correct, the model relies too heavily on inadequate data and is therefor fairly susceptible.
In their opinion, models tend to rely too extremely on data that does not have sufficient predictive power and do not account for parameter and estimation uncertainty.
To address these issues, their proposed Resampled Efficiency portfolios uses bootstrapping and are constructed by a conventional Monte Carlo algorithm that follows the following basic steps:
\begin{enumerate}
  \item Randomly sample means and covariance matrix of returns, with the center of these estimations being the regular values used in conventional mean-variance optimizations
  \item Calculate the mean-variance efficient frontier based on these sampled risk and return estimates
  \item Do this repeatedly to construct a big data set
  \item Average portfolio weights from all these generated data points
\end{enumerate}
The scholars show, that the resulting portfolio is more conservative and underperforms in-sample; but outperforms out-of-sample. Optionally, additional constrains can be added.\\
 