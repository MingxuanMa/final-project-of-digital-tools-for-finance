\section{Literature Review}
As mentioned above, the sensitivity of portfolio optimization to its input parameters is one of the key challenges of mean-variance theory in practice. In practice, out-of-sample validity is often overseen to the advantage of in-sample fit and optimal choices are by construction only the best choices in a given sample. Until today, practitioners often estimate constant means and covariances from historical data and assume them to be the correct values, which is obviously not true out-of-sample. Allowing for parameter uncertainty is, therefore, important, to allow for wrong input parameters and avoid over-fitting.\\
Puzzled by practitioners affection for outdated methods, academics proposed various sophisticated approaches to tackle this problem and tried to shift out-of-sample results into the focus (see \cite{albrecher_runggaldier_schachermayer_2009} for a detailed treatment). However, only \citeauthor{michaud_2007}'s simplistic resampling approach made it to somewhat an industry standard and will therefore be briefly discussed in the following, followed by a treatment of this work's main reference, \citeauthor{anderson_cheng_2016}'s \citetitle{anderson_cheng_2016} and some alternative methods like the ones proposed by \cite{ceria_stubbs_2006} and \cite{tütüncü_koenig_2004}. However, to start with, we will briefly discuss the origin of mean-variance optimization on the basis of \citeauthor{markowitz_1952}'s seminal work.

\import{./Sections/}{firstpaper}

\import{./Sections/}{secondpaper}

\import{./Sections/}{mainpaper}

\import{./Sections/}{thirdpaper}

\import{./Sections/}{fourthpaper}
