\section{Conclusion}
In this paper we look at different methods to allocate assets in a portfolio that go beyond simple Markovitz portfolio optimization. These robust approaches aim to be more stable in out-of-sample performance by accounting for real world uncertainty and frictions. Out of all the proposed methods we chose to replicate \cite{anderson_cheng_2016} approach that introduces a Bayesian probability framework. In each time period a new model is born and all probabilities of all previous models are updated to account for this new information. This information then gets aggregated to allocate the assets accordingly.\\
We compare this model to a simple Markovitz portfolio strategy and an equal-weight strategy. We find that depending on the time frame and performance metric, different models perform best. During higher volatility periods the Bayesian approach achieves the best results, while under more stable conditions the equally weighted model is superior. Therefore, in regard to our research question, whether \citetitle{anderson_cheng_2016} yield superior returns in an environment of extreme financial markets, we can conclude that this holds true based on our focus on sharpe ratios, max drawdowns and returns, even though the strategy was outperformed otherwise.
We look at sharpe ratio, max drawdown and return in a timeframe from 2005 until 2020 to evaluate these models. \\
While no clear winner emerged it is interesting to note that the Bayesian model might be a good method in more volatile times such as recessions.\\ Potential extensions would be the inclusion of GARCH specifications to model time varying volatility in the variance-covariance estimations, a combination of \citeauthor{anderson_cheng_2016}'s and \citeauthor{michaud_2007}'s approach, wherby the \citetitle{anderson_cheng_2016} technique is enforced by repeatedly bootstrapping past observations.
