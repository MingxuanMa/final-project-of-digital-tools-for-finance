\section{Introduction}
Asset management; and the broader discipline of investing, brought up many rich and talented men and women in the past, and made even more of the very same unlucky and poor. Their strategies and approaches were as diverse as the academic backgrounds they stemmed from and most had glorious but also less lucrative times.
Some of the most renowned are probably Warren Buffet, following the doctrines of his and Mr Graham's school; George Soros, taking a macro-driven approach; Peter Lynch, that focuses more on fundamentals; Robert Levine and his strong-horse method; David Swensen, that invests in alternative assets; or the brains behind the Renaissance Medallion Fund, that do algorithmic trading.\\
However, regardless of which strategy investors ultimately follow, probably all of them can agree on the benefits of diversification and the importance of asset allocation. Since the second half of the twentieth century, the economic theory underlying portfolio choice is well understood. Thanks to pioneering work like the one of \cite{markowitz_1952} and \cite{tobin_1969}, mean-variance optimized portfolios became the norm. This was further improved to settings with uncertainty by \cite{merton_1969} and dynamic settings by \cite{samuelson_1969}. While the discipline evolved further into multiple directions such as index funds in the 1970s; factor investing; and smart beta strategies in recent years, this basic dogma remained.\\
While in theory an easy endeavour, optimal asset allocation is sensitive to the true parameters in terms of covariance and conventional static approaches unsuitable to capture the complex and dynamic real-world. Despite these well-known pitfalls and shortcomings, little advances have made it into the investment practice of asset asset managers. Therefore, this paper aims to illustrate the approach for robust asset allocation suggested by \cite{anderson_cheng_2016} and replicate their strategy with real stock data from an extended period, reaching over fifteen years and thereby, covering two major equity sell-offs.