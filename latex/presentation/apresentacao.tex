\documentclass{beamer}

\mode<presentation>{
\usetheme{Dresden}
\setbeamercovered{transparent}
\usecolortheme{lsc}
}

\mode<handout>{
  % tema simples para ser impresso
  \usepackage[bar]{beamerthemetree}
  % Colocando um fundo cinza quando for gerar transparências para serem impressas
  % mais de uma transparência por página
  \beamertemplatesolidbackgroundcolor{black!5}
}

\usepackage{amsmath,amssymb}
\usepackage[brazil]{varioref}
\usepackage[english,brazil]{babel}
\usepackage[utf8]{inputenc}
%\usepackage[latin1]{inputenc}
\usepackage{graphicx}
\usepackage{listings}
\usepackage{url}
\usepackage{colortbl}

\beamertemplatetransparentcovereddynamic

\title[Quantitative Asset Management]{Final Project of Digital Tools for Finance}
\author[Mingxuan Ma, Yuke Xing, Yiyi Jiang]{%
  Autor: \\
  Mingxuan Ma, Yuke Xing, Yiyi Jiang }
  \institute[UZH]{%
     University of Zurich}

% Se comentar a linha abaixo, irá aparecer a data quando foi compilada a apresentação  
\date{10/12/2020}

\pgfdeclareimage[height=0.8cm]{inf}{figs/uzh_logo.jpg}
% pode-se colocar o LOGO assim

\pgfuseimage{inf}

\AtBeginSection[]{
  \begin{frame}<beamer>
    \frametitle{Contents}
    \tableofcontents[currentsection,currentsubsection]
  \end{frame}
}

\begin{document}

\begin{frame}
\titlepage
\end{frame}

\begin{frame}
\frametitle{Contents}
\tableofcontents
\end{frame}


\section{Introduction}

\frame{
    \frametitle{Introduction}
    \begin{itemize}   
       \item Our project is a replication of paper:
       \begin{itemize}
           \item Anderson E W, Cheng A R. Robust bayesian portfolio choices[J]. The Review of Financial Studies, 2016, 29(5): 1330-1375. 
       \end{itemize}
       
       \item Replicate the robust Bayesian model and compare its performance.
       \begin{itemize}
       	   \item with equally weighted model and standard Markowitz model on S&P 500 daily stock return from 2005 to 2020
       \end{itemize}
   \end{itemize}
}


\section{Literature Review}
\frame{
    \frametitle{Literature Review}
    \begin{itemize}   
       \item Modern Portfolio Theory 
       \item Resampling
       \item ‘Robust Bayesian Portfolio Choices’
       \item ‘Incorporating estimation errors into portfolio selection: Robust portfolio construction’
       \item ‘Robust Asset Allocation’
   \end{itemize}
}

\section{Research Question}
\frame{
    \frametitle{Research Question}
   Can ‘Robust Bayesian Portfolio Choices’ yield superior returns in an environment of extreme financial markets?
}


\section{Theoretical Framework}
\frame{
    \frametitle{Theoretical Framework}
   \begin{itemize}   
       \item Initialization
       \item Updating parameters in each t
       \item Updating probabilities for each model m in each t
       \item Allocate assets in each t
   \end{itemize}
}


\frame{
    \frametitle{Initialization}
    In period 1 we initialize mean, covariance and model- and scaling parameters.
   \begin{align*}
        \bar\mu_{t-1} &= \frac{1}{n}\sum_{i=1}^{n}(\frac{1}{t-1}\sum_{s=1}^{t-1} R_{i,s})\\
        \bar \lambda_{t-1} &= \frac{1}{n}\sum_{i=1}^{n}(\frac{1}{t-2}\sum_{s=1}^{t-1}(R_{i,s}-\bar\mu_{i,t-1})^2)\\
        \kappa_{t,t-1} &= 1,  \delta_{t,t-1} = 1, \tau = 4 \text{ and } \alpha = 1
    \end{align*}
}

\frame{
    \frametitle{Updating parameters in each t}
    At each point in time the mean and covariance matrix for all existing previous models is updated, using the newly observed excess returns $R_t$.
    \begin{align*}
        \mu_{m,t} &= \frac{\kappa_{m,t-1}\mu_{m,t-1} + R_t}{\kappa_{m,t}}\\
         \Sigma_{m,t} &= \frac{\delta_{m,t-1}\kappa_{m,t}\Sigma_{m,t-1}+\kappa_{m,t-1}(R_t - \mu_{m,t-1} )(R_t - \mu_{m,t-1})'}{\delta_{m,t}\kappa_{m,t}}
    \end{align*}
    With $\kappa_{m,t} = \kappa_{m,t-1} + 1$ and $\delta_{m,t} =\delta_{m,t-1} + 1 $, which can be viewed as scaling parameters that define the weight put on more recent inputs.
}


\frame{
    \frametitle{Updating probabilities for each model m in each t}
        \begin{align*}
            P_t(m | F_t) &= \frac{L(R_t | m, F_{t-1})P_t(m|F_{t-1})}{\sum_{m \in M_t} L(R_t | m, F_{t-1})P_t(m|F_{t-1})}
        \end{align*}
\\
        \begin{align*}
            L(R_t | m, F_{t-1}) &= \frac{ \kappa_{m,t-1}^{n/2} \det(\Lambda_{m,t-1}^{\nu_{m,t-1}/2}) \Gamma_n (\nu_{m,t-1}/2)}{\pi^{n/2} \kappa_{m,t}^{n/2} \det(\Lambda_{m,t}^{\nu_{m,t}/2}) \Gamma_n (\nu_{m,t-1}/2) }
        \end{align*}
\\
        \begin{align*}
            P_t(m | F_{t-1})=\left\{
                        \begin{array}{ll}
                          (1-\alpha) P_{t-1}(m |F_{t-1} + \alpha [\sum_{q=1}^{m} (\frac{1}{t-q+1}) P_{t-1}(q | F_{t-1})]), &\text{if }m<t\\
                          \alpha[\sum_{q=1}^{t-1}(\frac{1}{t-q+1})P_{t-1}(q | F_{t-1})],  &\text{if } m=t
                        \end{array}
                      \right.
        \end{align*}

}

\frame{
    \frametitle{Allocate assets in each t}
    The optimal portfolio weights $\phi_t$ can be calculated as follows.
    \begin{align*}
        \phi_t &= \frac{1}{\theta}\hat{\Sigma}_t^{-1}\hat{\mu}_t &\text{where }\\
        \hat{\Sigma}_{t} &= V(R_{t+1} | F_t)= \sum_{m \in M_t} (\bar{\Sigma}_{m,t} + \mu_{m,t}\mu_{m,t}')'P_t(m|F_t)-\hat{\mu}_t \hat{\mu_t}' &\text{with }\\
        \bar{\Sigma}_{m,t} &= V(R_{t+1} |m, F_t) = \Big(\frac{1+\kappa_{m,t}}{\kappa_{m,t}}\Big) \Sigma_{m,t}& \text{and }\\
        \hat{\mu}_t & = E(R_{t+1} |F_t) = \sum_{m \in M_t} \mu_{m,t} P_t(m | F_t)
    \end{align*}
    Next periods excess returns are observed based on the portfolio created in this instance. 
}

\section{Empirical Analysis}
\frame{
    \frametitle{Empirical Analysis Results}
        \centering
        \pgfdeclareimage[height=5.5cm]{sharpe}{figs/sharpe.png}
    \pgfuseimage{sharpe}
    \\
    
     \begin{tiny}
            (sharpe-ratio at t+1 from October 2008 to March 2009)
    \end{tiny}
}

\frame{
    \frametitle{Empirical Analysis Results}
    \centering
        \pgfdeclareimage[height=5.5cm]{return}{figs/return.png}
    %TODO fig RM

    \pgfuseimage{return}
    \\
    
     \begin{tiny}
            (portfolio return at t+1 from October 2008 to March 2009)
    \end{tiny}

}
\frame{
    \frametitle{Empirical Analysis Results}
        \centering
    	\begin{columns}
\column{3.3cm}
         \pgfdeclareimage[width=4cm]{1}{figs/Figure_prior.png}
          \pgfuseimage{1}\\
          
\column{3.3cm}
  \pgfdeclareimage[width=4cm]{2}{figs/Figure_prior2.png}
  \pgfuseimage{2}\\

\column{3.3cm}
  \pgfdeclareimage[width=4cm]{3}{figs/Figure_prior3.png}
  \pgfuseimage{3}\\
 
\end{columns}
}


\section{Conclusion}
\frame{
    \frametitle{Conclusion}
   Our main finding is that, with a small portfolio (N=30) and large rolling window (T=100), the robust bayesian model outperforms the other two before 2012, and this outperforming effect vanishes after 2012 when the paper is submitted. Also, we finds an interesting phenonmenon: robust bayesian model dominates the other two regardless of portfolio size from 2008 September to 2009 March, which is the period of financial crisis. A possible explanation could be that during crisis period, investors put more weight on prior information. Since investors are more likely to sell a stock which has a negative return yesterday due to panic rather than rational analysis during crisis period, prior information plays a more important role.
}


\end{document}